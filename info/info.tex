\documentclass[10pt]{article}

\usepackage{fancyhdr}
\usepackage{extramarks}
\usepackage{amssymb}
\usepackage{enumerate}
\usepackage{amsmath}
\usepackage{graphicx}

%
% Basic Document Settings
%

\topmargin=-0.45in
\evensidemargin=0in
\oddsidemargin=0in
\textwidth=6.5in
\textheight=9.0in
\headsep=0.25in

\linespread{1.1}

\pagestyle{fancy}
\rhead{\firstxmark}
\lfoot{\lastxmark}
\cfoot{\thepage}

\renewcommand\headrulewidth{0.4pt}
\renewcommand\footrulewidth{0.4pt}

\setlength\parindent{0pt}

%
% Create Problem Sections
%

\begin{document}

  \section{Que son?}
    Coleccion de cero o mas elementos.
    Cada elemento tiene prioridad o valor.
  \section{Operaciones Basicas}
    \begin{enumerate}
      \item getMin()
        O(1)
      \item getMax()
        O(1)
      \item put(x)
        insert x into the DEPQ
        O(log n)
      \item removeMin()
        O(log n)
      \item removeMax()
        O(log n)
    \end{enumerate}
    \textbf{Muchas estruturas de datos han sido propuestas para interpretar un DEPQ}
    \section{Implementaciones}
    \subsection{Symmetric Min-Max Heaps}
      \textbf{Creadores:} \\
      Arvind y Pandu Rangan
      Es un arbol binario completo, T. T cumple con lo siguiente:
      \begin{itemize}
        \item La raiz es vacia.
        \item $\forall v \in V(T), v tiene hermano derecho$, el elemento en v es menor o igual al de su hermano derecho.
        \item $\forall v \in V(T), v tiene abuelo$, el elemento en el hijo izquierdo del abuelo es menor o igual al elemento en v.
        \item $\forall v \in V(T), v tiene abuelo$, el elemento en el hijo derecho del abuelo es mayor o igual al elemento en v.
      \end{itemize}

      \textbf{Observaciones:}
      \begin{itemize}
        \item El numero de nodos es n+1.
      \end{itemize}

    


\end{document}

